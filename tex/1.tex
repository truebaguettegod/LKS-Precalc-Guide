\chapter{Sequences and Series}
\section{Definitions}
\begin{definition}
    A \emph{sequence} is a list of things, usually numbers. Each of the things in a sequence are called \emph{terms}.
\end{definition}
\begin{example}
    The sequence of the first $10$ positive even integers is
    \[\{2,4,\dots,18,20\}.\]
\end{example}
When we want to talk about sequences, we want to be able to label each of the terms in the sequence. Even though we can just refer to the first thing in a sequence as ``the first thing in a sequence'', it's very unwieldy to write
\[\text{the first thing in the sequence above}=2\]
every time we want to talk about the first thing in a sequence.

Instead, mathematicians use subscripts to refer to these elements. For example, we have that
\[a_1=2.\]
Here, the letter $a$ is just the name of our sequence, and the number $1$ means that it is the first thing in our sequence. We can treat $a_1$ as any other variable.

When we introduce a sequence, we can say that our sequence is $a_n$, $\{a\}$, $\{a_n\}$, or even just $a$. For example, we can alternatively say that
\begin{example}
    Let $a_n$ be the sequence of the first $10$ positive even integers, or
    \[a_n=\{2,4,\dots,18,20\}.\]
\end{example}
There also exist \emph{infinite} sequences. What this means is that the sequence has no end, compared to a \emph{finite} sequence which does have an end. Our example above is a finite sequence, since we can clearly see that it ends. The following sequence, on the other hand, does not end:
\begin{example}
    An example of an infinite sequence is the sequence of all positive even integers, or
    \[a_n=\{2,4,6,\dots\}.\]
    Notice that here, we cannot identify a concrete "last term". Even though the last term listed out is $6$, the ellipsis signify that there is more; in fact, infinitely more. Unlike with the earlier example where we used ellipsis because we're too lazy to list out all of the terms, here we do so because it is impossible to actually list out all of the terms.
\end{example}
A key thing to note is that sequences don't necessarily have any pattern. For example,
\[a_n=\{1,4,3,4,12,10,7,-3,\pi\}\] 
is a sequence. However, since these sequences are very random, we can't really identify special properties about them, so we won't discuss them very much.

On the other hand, a series is the sum of the things in a sequence. We may rigorously define it as follows:
\begin{definition}
    The \emph{series} $S$ of a sequence $a_n$ is defined as
    \[S=a_1+a_2+\dots\]
\end{definition}
\begin{example}
    Earlier, we had the example of the sequence $a_n=\{2,4,\dots,18,20\}$. Then, the series of this sequence is
    \[S=2+4+6+\dots+18+20,\]
    which calculates to be $110$.
\end{example}
Sometimes, we would like to only add up some of the elements of a sequence. We can define it as such:
\begin{definition}[Partial Sums]
    For a sequence $a_n$, the \emph{$n$th partial sum} is
    \[a_1+a_2+\dots+a_n.\]
    In other words, this is the sum of the first $n$ terms.
\end{definition}
\begin{exercise}
    Find the $6$th partial sum of the Fibonacci sequence
    \[a_n=\{0,1,1,2,3,5,8,\dots\}.\]
\end{exercise}
Later on in the section, we will explore special types of sequences which have special ways to calculate their partial sums.

\section{Explicit and Recursive Definitions}

There are many ways to define the terms of a sequence. For example, we can \emph{explicitly} define the terms of a sequence. Here, we explicitly tell the reader what the $n$th term exactly is.
\begin{example}
    The sequence of positive even integers $a_n$ is defined as
    \[a_n=2n.\]
    This is equivalent to saying "the $n$th term of our sequence can be written as $2n$".
\end{example}
Another way to define the terms of a sequence is \emph{recursively}. This is done by telling the reader how to compute each term of the sequence based on the previous terms of the sequence.
\begin{example}
    The sequence of positive even integers $a_n$ is defined with $a_1=2$, and
    \[a_n=a_{n-1}+2.\]
    This is equivalent to saying "you can get the next term of the sequence by adding $2$ to the previous one".
\end{example}
Initially, recursive definitions may seem incredibly useless, however examples like the Fibonacci sequence show that they make for much nicer presentations of sequences.
\begin{example}
    The famous Fibonacci sequence $F_n$ is defined with $F_0=0$, $F_1=1$, and for all other terms, that
    \[F_n=F_{n-1}+F_{n-2}.\]
    This is equivalent to saying "the $n$th term of the Fibonacci sequence is the sum of the terms before it". 
\end{example}
\begin{remark}
    Although there exist explicit definitions of the Fibonacci sequence, they are quite ugly and hard to work with, and it is best not to venture off in that direction until later. 
\end{remark}
The most important thing to remember for recursive definitions is \textbf{ALWAYS DEFINE THE FIRST TERMS OF A RECURSIVE SEQUENCE}. It is impossible for the reader to calculate the rest of the terms of your sequence if you don't explicitly define the first terms. Otherwise, it's like trying to knock dominoes down: even if the dominoes are set up to knock each subsequent one down, if you don't knock down the first one, the others will not fall.
\begin{exercise}
    Find the explicit and recursive definitions for the sequence 
    \[\{1, 2, 4, 8, 16, 32, 64,\dots\}\]
\end{exercise}
\begin{exercise}
    Find the recursive definition for the sequence
    \[\{-2, 3, 1, 4, 5, 9, 14,\dots\}\]
\end{exercise}

\section{Arithmetic Sequences}
The first special type of sequence is an arithmetic sequence. This is where any two consecutive terms have the same common difference. This brings us to the general definition of an arithmetic sequence.
\begin{definition}[The General Arithmetic Sequence]
    The general arithmetic sequence $A_n$ for two arbitrary constants $a, d$ is 
    \[A_n=\{a, a+d, a+2d, \dots, a+d(n-1)\}.\]
    We denote $d$ as the common difference.
    The explicit definition of that sequence is
    \[A_n=a+d(n-1).\]
    The recursive definition of that sequence is
    \[A_n=A_{n-1}+d\]
    \[A_n=a.\]
\end{definition}
The common difference can easily be calculated by subtracting any two adjacent terms of an arithmetic sequence, as it is always constant. 
Extending that, 
\[d=\frac{A_k-A_n}{k-n}\]
since there are $k-n$ differences between $A_k$ and $A_n$.
\subsection{The Arithmetic Series Formula}
We propose the following:
\begin{lemma}[Sum of Arithmetic Series]
    The sum of $\{a, a+d, a+2d, \dots, a+d(n-1)\}$ is 
    \[\frac{n(2a+d(n-1))}{2}\]
\end{lemma}
Now let's derive the arithmetic series formula. Here's a cool story that my parents have been teaching me since I was a child. It has been translated for your purposes.
\begin{remark}[Storytime]
    There once was a famous mathematician called Gauss. One day, his teacher gave his entire class a tricky problem: find the sum of the first $100$ integers. The whole class got down and started adding, frustrated. However, Gauss opted to avoid the grueling summation and decided to match each integer with its partner on the other side of the set. That is, he paired $1$ and $100$, $2$ and $99$, etc. Quickly, he found $50$ pairs of $101$, leading him to find $5050$ while the rest of the class was still stupidly chipping away at the problem.
\end{remark}
We can apply that thinking and write our proof.
\begin{proof}
    \[S=a+(a+d)+(a+2d)+\dots+(a+d(n-2))+(a+d(n-1))\]
    \[2S=(2a+d(n-1))+(2a+d(n-1))+(2a+d(n-1))+\dots+(2a+d(n-1))+(2a+d(n-1))\]
    We see $2a+d(n-1)$ appears $n$ times, so we get
    \[S=\frac{n(2a+d(n-1))}{2}\]
\end{proof}
Another way to think of this formula is 
\[\frac{n(First+Last)}{2}\]
This reason why should be pretty trivial. Lastly, we will finish this section off with a saying in Chinese that always helped me remember the formula
\begin{quote}
    shou xiang jia mo xiang cheng yu xiang shu chu yu er
\end{quote}
It simply means "First plus Last times occurences divided by two".
\begin{problem}
    Find $19+25+31+37+\dots+79+85+91+97$.
\end{problem}
\begin{problem}
    For an arithmetic sequence $S_n=\{a, a+d, a+2d,\dots,a+(n-2)d,a+(n-1)d\}$, find a formula for $S_k+S_{k+1}+S_{k+2}+\dots+S_{l-1}+S_l$ for arbitrary values $k$, $l$ such that $k\leq l$.
\end{problem}
\section{Geometric Sequences}
The second type of special sequence is a geometric sequence. This is where any two consecutive terms have the same common ratio. This leads us to our general geometric series.
\begin{definition}
    The general geometric sequence $G_n$ for two arbitrary constants $a, r$ is 
    \[G_n=\{a, ar, ar^2, \dots, ar^{n-1}\}.\]
    We denote $r$ as the common ratio.
    The explicit definition of that sequence is
    \[G_n=ar^n.\]
    The recursive definition of that sequence is
    \[G_n=G_{n-1}\cdot r\]
    \[G_n=a.\]
\end{definition}
The common ratio can be calculated by dividing two consecutive terms.

Similar to an arithemtic series, 
\[\frac{\frac{G_k}{G_n}}{k-n}\]
Similarily, the thinking goes that there are $k-n$ ratios between $G_k$ and $G_n$.
\subsection{The Geometric Series Formula}
We propose the following:
\begin{lemma}
    The sum of $\{a, ar, ar^2,\dots,ar^{n-1}\}$ is 
    \[\frac{a(r^n-1)}{r-1}\]
\end{lemma}
Sadly, there is no story to derive this formula. But there is a really interesting connection between this and polynomials. It just so happens that 
\[\frac{x^n-1}{n-1}=1+x+x^2+x^3+\dots+x^{n-1}\]
due to polynomial division (you can try this yourself with synthetic division). This is one way to derive the geometric series formula. Or you can do it the "official" way.
\begin{proof}
    \[S=a+ar+ar^2+\dots+ar^{n-1}\]
    \[S\cdot r=ar+ar^2+ar^3+\dots+ar^n\]
    \[Sr-S=ar^n-a\]
    \[S(r-1)=a(r^n-1)\]
    \[S=\frac{a(r^n-1)}{r-1}\]
\end{proof}
\begin{problem}
    Compute
    \[\frac{5}{12}+\frac{5}{3}+\frac{20}{3}+\dots+\frac{800}{3}\]
\end{problem}
\section{Infinite Geometric Series}
One funny (and really cool) thing we can do with geometric series that we can't do with arithmetic series is infinite geometric series. First, we need to define convergence and divergence. Convergence is when the sum of a series starts to approach, or \textbf{converge} onto a point as more terms are added. Divergence is when the sum of a series doesn't approach anything, and in fact moves farther away from $0$ on the number line.
Geometric series, however, have the potential for values to approach $0$ as $n$ grows larger if the common ratio is less than $1$. Therefore, the series has the ability to converge onto a value.
A famous problem is the dichotomy problem, which details the story of someone trying to get from point $A$ to point $B$. Each turn, they move half the remaining distance to point $B$. Will they ever reach point $B$? It turns out, sorta. 
\subsection{Infinite Geometric Series Formula}
We really want to find a value for our dichotomy problem, but to do so, we have to eliminate the infinite terms. Luckily, since it's infinite, we can multiply our series by $2$. 
\[S=\frac{1}{2}+\frac{1}{4}+\frac{1}{8}+\dots\]
\[2S=1+\frac{1}{2}+\frac{1}{4}+\dots\]
\[2S-S=S=1.\]
Now we can apply this thinking and obtain our infinite geometric series formula.
\begin{lemma}
    The sum of $\{a+ar+ar^2+ar^3+\dots\}$ is
    \[\frac{a}{1-r}\]
    if and only if $|r|<1$.
\end{lemma}
\begin{proof}
    \[S=a+ar+ar^2+ar^3+\dots\]
    \[S\cdot r=ar+ar^2+ar^3+ar^4+\dots\]
    \[S-S\cdot r=a\]
    \[S=\frac{a}{1-r}\]
    Obviously, if $r\geq 1$, the terms can never converge, so the restriction that $|r|<1$ applies.
\end{proof}
If we decide to violate this restriction, we get really interesting values. This is the study of infinite partial sums. Though it technically isn't defined, we can apply this to some series with common ratio $\geq 1$ and get some pretty strange answers. Though it isn't an infinite geometric series, the Ramanujan Summation tells us
\[1+2+3+4+\dots=-\frac{1}{12}.\]
This is obviously false. Or is it? Srinivas Ramanjuan developed a very interesting derivation, which is explained in the subappendices.
\begin{problem}
    Compute
    \[\frac{1}{3}+\frac{3}{6}+\frac{9}{12}+\frac{27}{24}+\dots\]
\end{problem}
\begin{problem}[\textbf{Challenge}]
    Compute
    \[\frac{1}{2}+\frac{3}{4}+\frac{5}{8}+\frac{7}{16}+\dots\]
\end{problem}
\section{Sigma Notation}
As a mathematician, I love efficiency. One thing that cosntanlty annoys me is writing out a summation, like $a+ar+ar^2+ar^3+ar^4+\dots$. Don't you wish there could be a quicker way of writing this? Well, luckily there is. Welcome to the wonderful world of sigma!
Sigma notation is an efficient way to write out summations or products (which is technically called product notation, but we're putting them in the same section).
\[\sum_{k=a}^{n}f(k)\]
This is sigma. If you know coding, think of it as a for loop. $k$ is the index of this summation, $a$ is the start, and $n$ is the stop. That is, it will sum $f(k)$ for every value of $k$ from $a$ to $n$, inclusive. $k$ will only increase by $1$ each time. 
\begin{example}[Sum of All Natural Numbers]
    \[\sum_{k=1}^{\infty}k=1+2+3+4+\dots\]
    The start is $1$, so we start with $1$. Then we add $2$. Then $3$. This continues for eternity, since you can have the stop be infinity. I suppose in that case, the summation just doesn't stop. 
\end{example}
Product notation is very similar. Just replace $\sum$ with $\prod$, and instead of summing terms, multiply terms.
\begin{example}[n!]
    \[\prod_{k=1}^{n}k=1\cdot 2\cdot 3\cdot \dots \cdot n=n!\]
\end{example}
You are allowed to nest summations, such as 
\[\sum_{i=1}^{n}\sum_{j=1}^{i}j=(1+2+3+\dots)+(2+3+4+\dots)+\dots\]
However, something you are not allowed to do is have the index move backwards. This is not standard:
\[\sum_{k=10}^{5}k.\]
\begin{problem}
    Write $(1\cdot2)+(2\cdot 3)+(3\cdot 4)+\dots$ using sigma and product notation.
\end{problem}
\section{Interest}
- explain APR/Yield
- explain how sequences+series tie into it
- problem 

\begin{subappendices}
\section{Bonus: Sums of $n^k$ powers}
    - formulas, how to derive next
\section{Explicit Formulas for Linear Recurrences}
\section{The Ramanujan Summation}
\end{subappendices}