\documentclass[11pt]{scrartcl}
\usepackage[sexy]{evan}
\usepackage{appendix}
\setcounter{section}{-1}

\begin{document}
\title{Honors Precalculus: Yet another Survival Guide}
\author{Timothy Dong}
\date{Lakeside School}
\maketitle

\tableofcontents

\newpage{}
\section{Introduction}
Welcome to the Honors Precalculus Survival Guide! This is a document containing a collection
of concepts, formulas, and topics that will be covered in the M420 Honors
Preacalculus course (at least in the 2023-24 curriculum). Each chapter, 
or "Unit," will run through all the ideas discussed, and provide a relatively
brief and easy to understand explanation for some ideas, as well as calculations
and derivations of formulas paired with written logic and reasoning. There
will also be "bonus" chapters on subjects that won't really be tested on,
but are cool and helpful to know.
\\\\
The guide inevitably prioritizes effiency, but I want to emphasize the utmost
importance of understanding core concepts rather than the memorization of
formulas. As such, I strongly suggest you think about the "why" behind
every step because it will help reinforce your understanding of topics 
through intuitive explanations and the connection of ideas.
\\\\
I decided to write the Precalc Guide because there was a lot of demand for
a guide for this class specifically, and because of the notorious difficulty 
fo this class. I'll just say upfront: Honors Precalc isn't hard. That is,
if you pay good attention and try to understand the "why" rather than the
"what," everything falls into place. But I still wanted to make this to hopefully
break the "harder" parts of this class into easier pieces and alleviate some
stress behind this class.
\\\\
In the end though, this was heavily inspired by Michael Y. '24, the author
of the original Lakeside math course guide titled "Multivariable Calculus: A Survival Guide."
Without his contributions to Lakeside math, this would not have been possible.
I highly suggest checking his document out for all your Multi needs.
\\\\
Because I am "by no means an expert at [pre]calculus," to quote Michael,
there will inevitably be errors and confusing parts. If anything doesn't
add up (get it?), feel free to let me know. 
\\\\
In closing, I wish you all good luck on your journey through Precalculus. The
document is titled such because it covers Honors Precalculus topics, but anyone
is free to use this as a reference. However, Honors or not, Precalculus is
not as terrifying as it seems. Like I always say, "As long as you study, you'll
be fine." Prioritize sleep, pay attnetion, and enjoy the wild ride. It's a
fun class that will lead you to life-changing discoveries as long as you just stay curious. 

\newpage{}
\section{Unit 1: Sequences and Series}
- intro to what is a sequence vs series
- explain partial sums
\subsection{Explicit vs. Recursive Definitions}
- cover explicit Definitions
- cover recursive def 
- problem
\subsection{Arithmetic Sequences}
- explain arithmetic series
- formulas+derivations
- problem
\subsection{Geometric Sequences}
- explain geometric series
- formulas+derivations
- problem
\subsection{Sigma Notation}
- explain the importance of Sigma
- explain how to use sigma
- problem
\subsection{Infinite Series}
- explain the importance of Sigma
- explain divergence+convergence
- problem
\subsection{Interest}
- explain APR/Yield
- explain how sequences+series tie into it
- problem
\begin{subappendices}
    \subsection{Bonus: Sums of $n^k$ powers}
    - formulas, how to derive next
    - 
\end{subappendices}

\newpage{}
\section{Unit 2: Freaky Functions}
- what is a Function
- what is not a function
text

\newpage{}
\section{Unit 3: The Derivative}

text

\newpage{}
\section{Unit 4: Exponential and Logarithmic Functions}

text

\newpage{}
\section{Unit 5: Tricky Trigonometry}

text

\newpage{}
\section{Unit N: Topic 1}
Did you know that I will be talking about animals?
\subsection{Subtopic}
Did you know that fish have 0 legs?
\begin{theorem}[The Fish]
    Fish have 0 legs
\end{theorem}
\begin{proof}
    Because they can't walk, dummy 
\end{proof}
What else can fish do?
\begin{lemma}
    Fish also can't fly
\end{lemma}
That is for our purposes, true.
\begin{remark}
    Flying fish can fly, but they are a conspiracy
\end{remark}
\subsection{Subtopic 2}
Did you know that chickens have two legs?
\subsubsection{Subsubtopic 1}
Did you know that those chickens who have two legs also have beaks?
\begin{subappendices}
    \subsection{Bonus: Special Topic that isn't on tests}
    Did you know that horses have four legs? That's crazy.
\end{subappendices}

\appendix{}
\section{Appendix A: Something}

\section{Works Cited}

\section{Special Thanks}
I really want to thank \textbf{Raymond Z. '27} for massively helping me
and carrying my \LaTeX.

\paragraph{}

Michael Y. '24 for being a gigachad.


\end{document}